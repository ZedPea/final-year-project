\section{Background}
\subsection{Server}
The server will host all the game logic, and will take input from the client, update its internal state, and propagate the changes to all the other clients.
This section of the project will require managing a complex state, message passing and gracefully handling failure when network conditions cause clients to disconnect unexpectedly.

\tikzset{
    diagram/.style={
        rectangle, 
        rounded corners, 
        draw=black, very thick,
        text width=10em, 
        minimum height=3em, 
        text centered},
}
\vspace*{0.8cm}
\begin{center}
\begin{tikzpicture}[framed]
    \node[diagram, outer sep = 3pt] (server)     at (4, 0.0) {Server};
    \node[diagram, outer sep = 3pt] (client)     at (0,-5.5) {Client};
    \node[diagram, outer sep = 3pt] (ai)         at (8,-5.5) {AI};

    \node[diagram, minimum height=1em, outer sep=3pt]    (hsqml)   at (0, -7.1) {HsQML};
    \node[diagram, minimum height=1em, outer sep=3pt]  (qtquick)   at (0, -8.5) {QtQuick};
    \node[diagram, minimum height=1em, outer sep=3pt]  (network)   at (4, -1.5) {Network};
    \node[diagram, minimum height=1em, outer sep=3pt] (network2)   at (0, -3.8) {Network};
    \node[diagram, minimum height=1em, outer sep=3pt] (network3)   at (8, -3.8) {Network};

    \draw[thick, <->]   (server.south) -- (network.north);
    \draw[thick, <->]   (client.north) -- (network2.south);
    \draw[thick, <->] (network2.north) -- (network.south west);
    \draw[thick, <->] (network3.north) -- (network.south east);
    \draw[thick, <->]   (client.south) -- (hsqml.north);
    \draw[thick, <->]    (hsqml.south) -- (qtquick);
    \draw[thick, <->] (network3.south) -- (ai.north);
\end{tikzpicture}
\end{center}
\vspace*{0.5cm}

\subsection{Client}
The client code will be relatively simple, just taking simple inputs from the user, and passing it to the server, and interpreting the servers responses.
It will, however, take significant effort to make a good looking GUI, potentially featuring animations and different themes.
Due to Haskell having immutable state by default, which is poor for many graphics libraries, I have decided to use the QtQuick graphics library, which uses JSON to describe a UI\@.
The interaction between Haskell and QtQuick is minimal and easy to use.
\subsection{AI}
The AI could potentially be a significant part of this project, based on time constraints.
There have been lots of papers on this subject which will allow for multiple different approaches.
A rule based AI\cite{watson2008} could be a good starting point, after which more advanced techniques could be implemented, such as neural networks\cite{teofilo2011}, and statistical models\cite{billings2002}.
