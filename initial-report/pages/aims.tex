\section{Aims \& Objectives}

\begin{itemize}
\item Investigating the capability of Haskell, a functional programming language, in programming networks, stateful code, and AI
\item Creating a rich GUI with a high level toolkit, allowing modular replacement of assets at users request
\item Investigating a number of different Artificial Intelligence programming paradigms for poker
\end{itemize}
Motivations behind the decision to use Haskell include:
\begin{itemize}
\item A strong type system, allowing rapid development and refactoring
\item Functional programming, which encourages terse, expressive code
\item Simple and safe multi threading, due to immutable by default data types, requiring very little change to application code to take advantage of multi-core processes
\item High level support for managing complex state and other side effects, using monads\cite{benton2000}
\item High level support for updating and viewing nested state, using the ``lens'' library and lenses paradigm\cite{bohannon2006}
\item A rich library system allowing massive amounts of duplicate code to be avoided
\end{itemize}
Significant libraries to be used:
\begin{itemize}
\item transformers-This library provides a high level interface for managing stateful code, using monad transformers, and implicit parameters
\item lens-This library allows painless updating of deeply nested structures, creating very expressive and concise code
\item network-This library has a socket interface, and will be used for the client and server, offering lots of power, whilst having a very simple interface
\item hsqml-A Haskell interface to the QtQuick graphics library, which allows fast prototyping with JSON syntax
\item binary-A library for encoding arbitrary data types into a binary format for efficient message passing between client and server
\end{itemize}
