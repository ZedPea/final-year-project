\section{Evaluation}

\subsection{Further Work}
There are a few areas of the project that could be improved. One issue is that
when side pots are created, due to one player being all-in and unable to bet
anymore, this is not visually indicated. The other players cannot tell how
large the side pot or multiple side pots are without doing arithmetic manually,
which is not very user friendly. It is difficulty to display side pots well
without having a much more intensive graphics framework which allows for easy
placement and spacing out of graphical assets. Such a framework would be an
interesting avenue for exploration, as it would also allow a richer experience
to be created with the possibility for animations, such as when a card is
flipped, or chips are moved into the pot, or changing of graphical elements
on the fly, for example to let users choose different table and card themes.

Another weakness of the project is the networking is not very mature. A player
is given infinite time to bet, and if they disconnect mid bet, this will
cause the server to remain stuck waiting for them to respond. If they return
an invalid message, they will be folded as expected, but some method of timing
out players who disconnect or do not respond in time could be implemented.
This could be enhanced to allow users to sit out for a round, or be
automatically moved to sit out when they don't respond, allowing them to
rejoin the game when they regain internet connectivity or return from their
break. This would have to be handled to prevent users from sitting out when
they are required to pay the big or small blind, and then rejoining when they
can get a free set of cards, perhaps by making them only able to rejoin
the table when it is their turn to pay the big blind. In addition to this,
on the client side code would need to be developed to notify a user when they
have lost and regained connection, and to allow them to choose to sit out a 
round.

Another idea which builds on the previous idea is allowing users to join
tables in progress which are missing players. A system is already in process
to allow multiple games to be ongoing at once, and if the timing out solution
is applied, instead of new users having to wait for a whole new game to be
ready, they could instead be instantly seated at a table which is missing
a player. This would prevent games slowly losing players and delaying users
from playing the game. This system has already been nearly completed, and
would be trivial to implement once the timeout solution has been applied.

Another improvement on the user experience could be a lobby waiting screen.
Currently whilst a user is waiting for a game they remain in a command prompt
until the game is filled, with no feedback how many players are waiting or
how many more are needed to start a game. Only once the game starts does the
graphical user interface launch. If the network is not very active, a user
might be stuck on this screen for a while, wondering if the program is working
or not. A graphical user interface for a lobby waiting system could be
developed, allowing users to view how many people are online, how many
people are waiting for a game, and how many people are currently in a game.
This would also allow for users to easily select different tables, potentially
running different game modes.

One final user experience improvement could be balance storing. Currently,
a user starts with 1000 chips and regardless of if they gain or lose chips
in the game, when they start another game they are restored to 1000 chips.
If a database was utilised, we could allow users to have accounts to store
their balances, allowing them to rack up larger amounts of chips by winning
multiple times. This would have to be balanced by allowing tables with
different minimum bets, and max allowed balance, to prevent users with
large chips using their wealth advantage to bully smaller holders with constant
large raises.

In regards to the AI, this is currently implemented as just a simple rule
based algorithm, where it bets upon the calculated value of its hand and nothing
more. This could be extended by making the AI randomly attempt to bluff when
it has a poor hand, to prevent a player being able to counteract the AI's
play-style by simply folding whenever the AI makes large bets and matching when
the AI makes medium bets and the player has a good hand.

Another more ambitious upgrade would be to create the AI using machine learning
techniques, and exploring a powerful AI that is able to challenge the best
human players at poker performance. An initial version of this was attempted
using tensorflow and python, but was shortly abandoned to focus on other
features.

\subsection{Project Achievements}
The original goal for this project was to create a comprehensive software
solution to allow people to player poker, and a strong AI for the users to
practice their skills against.

The project has pivoted from its original goals somewhat. Whilst
developing the program, the choice of shuffle algorithms and random sources
was encountered, and was considered a valuable area to investigate. The
role of AI was pivoted from allowing users to practice their gameplay into
a system for automatically testing the software, and investigating the
effect of the previously mentioned algorithms and random sources. The goal
to create a strong AI with machine learning is still a very interesting one,
and could be explored in the future. In this way, the project has shifted
from a user orientated software to a more investigative role and has made
some interesting discoveries. The software is still perfectly usable to play
poker however, and furthermore the AI developed performs decently against
human players. This does allow the users to still attempt to practice their
skills against it, but without having a strong AI there is only so much they
can learn before they will be easily outperforming the AI\@.
