\section{Introduction}

%%%%%%%%%%%%%%%%%%%%%%%%%%%%%%%
\subsection{Project Environment}
%%%%%%%%%%%%%%%%%%%%%%%%%%%%%%%

Poker has become a very popular game to play both online, and in real life.
Many different companies offer online poker software, with the opportunity
to gamble real money with ease. Christian Capital Advisors reported that
online poker revenues were \$2.4 billion in 2005 \parencite{website:newsweek2005},
and the popular online poker vendor, PokerStars, were acquired for \$4.9 billion
in 2014. \parencite{website:heinter2014}

Whilst these large amounts of money are being gambled, very little has been
investigated into the security of the software used. \parencite{arkin1999}
shows how bad random number generation can be exploited to cheat in online
poker.

%%%%%%%%%%%%%%%%%%%%%%%%%%
\subsection{Project Goals}
%%%%%%%%%%%%%%%%%%%%%%%%%%

This report will set out some of the issues and pitfalls with developing an
online poker software suite, especially in relation to randomness and shuffling.
Solid testing tools will be looked at, including the use of AI to test rare
code paths rapidly. The reasons behind design choices and architectural choices, along with
evidence to support these will be detailed.

This report has six sections that will be covered, detailing each section of
the project. They are as follows.

\begin{itemize}
    \item Introduction - Introduces the project
    \item Aims and Objectives - Highlights what the project wishes to achieve
    \item Background - Describe the problem context and compare some technologies available
    \item Technical Development - Will discuss the design, testing, and implementation of the software
    \item Evaluation - A summary of the project progress, and what further work could be explored
    \item Conclusion - A final review of the project and what has been achieved
\end{itemize}

%%%%%%%%%%%%%%%%%%%%%
\subsection{Use Case}
%%%%%%%%%%%%%%%%%%%%%

A programmer or team of programmers can use this software as a reference
implementation when designing a poker game or similar gambling software. They
can experiment with the different shuffling algorithms and random sources
used to find those which provide a sufficient level of randomness and security
of the deal. They can utilise the automated testing facilities developed to
test their own software and ensure it is bug free, and finally, they can
use the developed network protocol to allow easy interoperability with both
their software and this software, for simple drop in support with alternative
clients and servers.
